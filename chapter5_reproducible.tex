\chpt{Reproducible research}\label{chpt:repro}

Nowadays, the concept of reproducible research is advocated by more and more researchers in the scientific community \citep{Peng:2011et, Sandve:2013gh}. The idea of reproducible research is that any computational results the researchers generate, such as numbers, figures, tables, etc., can be re-generated with minimal effort by themselves and other researchers. This is especially necessary for computational research. The computational analysis usually involves  multi-step data preprocessing and the statistical models and computational tools used in the analysis often involves many parameters.   The analysis procedure, statistical models and computational tools need to be validated by other researchers. 

One important idea in reproducible computational research is to use literate programming. Simply put, literate programming is putting the code with results and annotations together in just one file. In R, one can generate this type of file with RMarkdown (or knitr) package. In Python, one can do it with IPython Notebook. Since R is mainly used for the analysis in this dissertation, I chose RMarkdown as the tool for reproducible research. The output of RMarkdown reports can be PDF, Word or HTML format. HTML format can be distributed easily and researchers can view it in the regular web browser, all the analysis  in this dissertation are reported in HTML format. Figure~\ref{F61_Reproducible_Research} shows an example of the RMarkdown report in HTML format. The code and corresponding results such as figures and tables are integrated together in one report. The code along with the parameters and options are demonstrated above the results. Researchers who want to repeat the analysis can rerun the same code.



\begin{figure*}[p]
	\centering
	{\includegraphics[scale=0.4,trim=0 0 0 0,clip]{Figure/F61_Reproducible_Research.pdf}}
	\caption[An example of RMarkdown report]{An example of Rmarkdown report. The RMarkdown integrates R code, analysis output such as numbers, figures, tables, and annotation text. The analysis in the RMarkdown report is organized and indexed by an index table.  All the analysis included in this dissertation were implemented as RMarkdown reports.
	}
	\label{F61_Reproducible_Research}
\end{figure*}


To make sure that all the statistical models proposed in this dissertation can be easily applied by other researchers in their research, I implemented the statistical models in R packages and put them in Github (Figure~\ref{Github_MSSQ}). Github is freely accessible to public users and it is easy to install R package directly from Github. In this way, researchers do not need to implement the proposed statistical models by themselves. Users who identify bugs in the package or have any improvement suggestions can comment on the corresponding  package website. 

\begin{figure*}[p]
	\centering
	{\includegraphics[scale=0.4,trim=0 0 0 0,clip]{Figure/F62_Github_MSSQ1.pdf}
	\includegraphics[scale=0.4,trim=0 0 0 0,clip]{Figure/F63_Github_MSSQ2.pdf}
		
	}
	\caption[An example of my R package on Github]{An example of R package MSSQ on Github. Researchers who are interested in using this package can easily install it from Github. All the code are freely available to the public users. An instruction of the package is also included at the main website of this package. Another R package ZIBR introduced in this dissertation was also submitted to Github in a similar fashion.
	}
	\label{Github_MSSQ}
\end{figure*}


Some analysis are also implemented in interactive web applications with R Shiny (Figure~\ref{F64_Shiny}). Users can explore the analysis by simply changing certain parameters. The figure will then be automatically updated. This is especially useful in exploratory analysis that researchers want to investigate one type of analysis in different parameter settings.


\begin{figure*}[p]
	\centering
	{\includegraphics[scale=0.4,trim=0 0 0 0,clip]{Figure/F64_Shiny.pdf}
	}
	\caption[An illustration of interactive analysis of microbiome data]{An illustration of interactive analysis of microbiome data. Users can analyze and visualize the correlation between metabolites and bacterial taxa in different groups. The user can also filter out the taxa with low correlation. The figure will be automatically updated after users change the setting. This interactive web application was implemented with R Shinny. 
	}
	\label{F64_Shiny}
\end{figure*}